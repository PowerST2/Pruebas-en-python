Una partícula realiza un movimiento con trayectoria circular de $2$ m de radio, con aceleración angular de módulo $1.5$ rad/s$^{2}$. Si en cierto instante la magnitud de la aceleración normal es igual a la magnitud de la aceleración  tangencial, calcule aproximadamente (en m/s$^{2}$), el módulo de la aceleración de la partícula en dicho instante.
\ctd{1.8}{2.7}{4.2}{5.4}{6.5}
